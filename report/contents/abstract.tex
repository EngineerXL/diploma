\abstract % Структурный элемент: РЕФЕРАТ

\keywords{ЦИФРОВОЙ ДВОЙНИК, РЕЛЯЦИОННЫЕ БАЗЫ ДАННЫХ, ВРЕМЕННЫЕ РЯДЫ, СОРТИРОВКА, ДВА УКАЗАТЕЛЯ, ДЕРЕВЬЯ, ПОИСК В ГЛУБИНУ}

Объектом разработки в данной работе является часть цифрового двойника предприятия.

Цель работы --- разработать модуль, обеспечивающий управление структурой хранения временных рядов и данными сенсоров.

Основное содержание работы состояло в разработке алгоритма управления графом организационной структурой и объединения данных датчиков с разными частотами дискретизации.

Основным результатом работы является модуль управления временными рядами сигналов сложных технических систем на языке Python с использованием СУБД PostgreSQL и ClickHouse.

Данные результаты разработки предназначены для надёжного хранения данных датчиков и последующего использования при моделировании объекта и предиктивной аналитики.

Внедрение модуля позволяет автоматизировать сбор информации с сенсоров системы, тем самым упрощая создание цифрового двойника электростанции.
