\introduction % Структурный элемент: ВВЕДЕНИЕ

Актуальность темы данной работы связана с распространением цифровых двойников объектов и систем. Их создание позволяет моделировать отдельные процессы или объекты целиком, проводить тесты, анализировать полученные данные для подбора оптимальных параметров системы. Один из способов создания цифрового двойника — установка сенсоров и сбор данных с них. Полученную информацию необходимо систематизировать и хранить. На предприятии может быть очень много оборудования, поэтому нужно внедрять эффективный и надёжный модуль хранения данных датчиков.

Таким образом, выполненная работа актуальна и с теоретической, и с практической точек зрения.

Цель работы --- разработать модуль, обеспечивающий управление структурой хранения временных рядов и данными сенсоров. Для достижения поставленной цели в работе были решены следующие задачи:

\begin{itemize}
    \item спроектирована модель данных дерева организационной структуры предприятия;
    \item описаны способы взаимодействия: добавление, удаление и изменение вершин и рёбер дерева;
    \item спроектирована модель хранения временных рядов датчиков;
    \item изучены средства и технологии, которые будут применятся в ходе разработки программного продукта;
    \item реализован модуль управления графом организационной структурой и данными;
    \item разработан алгоритм объединения данных датчиков с разными частотами дискретизации;
    \item реализована генерация данных для таблиц датчиков, алгоритм получения наборов временных рядов;
    \item произведён тест производительности реализованного модуля, проведено сравнение двух алгоритмов хранения и считывания данных.
\end{itemize}

Для разбработки программы необходимо изучить инструменты и методы, решающие поставленные задачи. Работа основывается на следующих СУБД, библиотеках, технологиях и алгоритмах:

\begin{itemize}
    \item Python является основным языком программирования, который использовался при решении задач;
    \item FastAPI реализует веб-интерфейс для взаимодействия с модулем и базами данных, SwaggerUI визуализирует веб-интерфейс;
    \item SQLAlchemy позволяет работать с базами данных на основе объектно-ориентированного подхода;
    \item PostgreSQL обеспечивает хранение дерева организационной структуры предприятия и информации о датчиках;
    \item ClickHouse хранит большие объёмы данных, получаемые от сенсоров;
    \item Docker позволяет разворачивать и переносить изолированные контейнеры с базами данных;
    \item GraphViz визуализирует дерево организационной структуры;
    \item метод двух указателей используется для объединения таблиц датчиков с разными частотами дискретизации.
\end{itemize}

В результате выполнения работы был разработан модуль управления временными рядами и деревом организационной структуры предприятия, позволяющий генерировать и полученать наборы временных рядов, управлять, изменять, визуализировать граф организационной структуры, добавлять новые датчики и организационные единицы.

Результаты работы предназначены для автоматизации сбора данных данных с датчиков, установленных на предприятии. Собранные данные передаются в базу данных для последующего мониторинга, диагностики и аналитики, расчёта оптимальных параметров на предприятии.

Использование разработки позволяет ускорить процесс создания цифрового двойника системы, а так же сделать его более точным. Модуль обеспечивает надёжное хранение организационной структуры предприятия и быстрый доступ к данным сенсоров оборудования.