\section{ЦИФРОВЫЕ ДВОЙНИКИ КОМПАНИИ}

\subsection{Цифровые двойники, их применение и автоматизация создания}

Цифровой двойник --- это виртуальная копия объекта или системы, созданная на основе данных, полученных из реального мира. Цифровые двойники могут быть созданы для любого объекта или системы, от автомобилей до зданий и даже городов.

Они используются в различных сферах и областях, включая проектирование и строительство, управление городами и транспортом, энергетику и промышленность, медицину и многое другое. Цифровые двойники позволяют смоделировать объект или систему в виртуальной среде, чтобы оптимизировать его производительность, улучшить безопасность и снизить затраты.

Одним из наиболее распространенных применений цифровых двойников является проектирование и строительство зданий. Цифровые двойники зданий могут использоваться для оптимизации проектирования, улучшения эффективности энергопотребления и сокращения времени строительства. В промышленности цифровые двойники используются для оптимизации производственных процессов, улучшения качества продукции и снижения затрат на производство. Цифровые двойники могут помочь смоделировать производственную линию и определить оптимальные настройки оборудования для снижения изонса. Они также используются в медицине для создания виртуальных моделей пациентов. Это позволяет врачам более точно диагностировать и лечить заболевания, а также планировать сложные операции.

В целом, цифровые двойники становятся все более распространенными и играют важную роль в различных областях. Они помогают улучшить безопасность и экономическую эффективность объектов и систем, а также ускоряют процесс разработки и производства.

Для автоматизации создания цифровых двойников необходимы компьютерные алгоритмы и технологии, создающие копии объектов и систем. Такой подход позволяет не только ускорить процесс создания цифровых двойников, но и улучшить их точность и качество.

Одним из распространенных методов автоматизации создания цифровых двойников является использование программного обеспечения для моделирования. Это позволяет создавать точные виртуальные модели объектов и систем на основе данных, полученных из реального мира. Программное обеспечение может быть настроено для определения оптимальных параметров объекта или системы, а также для проведения различных симуляций и тестов.

Ещё одним методом является использование искусственного интеллекта и машинного обучения. Это позволяет создавать более точные и детальные модели объектов и систем, а также оптимизировать процесс создания цифровых двойников. Искусственный интеллект может быть настроен для обработки больших объемов данных и автоматического анализа полученной информации.

Автоматизация создания цифровых двойников имеет большое значение для различных отраслей энергетики, промышленности, медицины, строительства и других областей. Не менее важно собирать и хранить данные объекта для создания цифрового двойника.

Автоматизировать сбор данных с объекта можно с помощью различных сенсоров, таких как лазерные сканеры, фотокамеры, акселерометры, гироскопы и другие. Датчики могут быть установлены на объекте или системе и использоваться для удалённого сбора данных. Собранные данные затем обрабатываются и хранятся в базе данных.

Другим методом автоматизации сбора данных является использование систем мониторинга и диагностики. Эти системы могут быть установлены на объекте или системе и использоваться для непрерывного мониторинга и анализа различных параметров, таких как температура, давление, вибрация, электрические параметры и другие. Собранные данные затем передаются в базу данных для дальнейшей обработки и хранения.

Поступающих данных может быть очень много, поэтому важно быстро и эффективно сохранять их в базе данных. Потери этих данных недопустимы, так как могут привести к серьёзным последствиям при моделировании объекта или системы, для которой создаётся цифровой двойник.

\subsection{Этапы, результаты и сложности при создания цифровых двойников предприятия}

С ростом цифровизации в различных отраслях растёт и развитие цифровых двойников в промышленности и энергетике. По данным~\cite{Habr1} на 2021 год 18 ведущих мировых компаний уже используют цифровые двойники, а 24 тестируют их применение. В России к 2024 250 ведущих компаний планируют внедрить эту технологию.

Создание цифровых двойников --- достаточно сложный процесс, который можно разделить на следующие этапы~\cite{Habr2}:

\begin{itemize}
    \item обследование начинается с изучения нормативных документов, карт и инструкций по экплуатации, часто информация не закреплена или исполнятся не строго по бумагам, поэтому необходимо проводить интервью с работниками для воспроизведения процессов на предприятии максимально близким к реальным;
    \item разработка обычно является самым долгим и трудным этапом, необходимо создать инфраструктуру предприятия в виртуальной среде, разбработать программные продукты для взаимодействия;
    \item валидация заключается в проверке точности модели на основе данных работы предприятия в прошлом, чем ниже расхождение модели с реальной работой, тем выше качество модели;
    \item эксплуатация --- цифровой двойник внедряется в работу и используется для решений актуальных задач предприятия.
\end{itemize}

В зависимости от степени цифровизации предприятия срок создания цифрового двойника варьируется от трёх месяцев до года. Чем выше готовность бизнеса к цифровизации, тем проще проходят описанные выше этапы. Рассмотрим несколько примеров из~\cite{Habr1}, как цифровые двойники улучшили работу на предприятиях.

Морской порт на юге России нуждался в оптимизации распределения нагрузки и планирования смен. Все данные передавались на бумаге, их вручную обрабатывал оператор. Такой подход сильно зависит от человеческого фактора --- вероятность ошибки и скорость работы зависит от оператора. Создание цифрового двойника позволило лучше составлять планы, что повысило среднесуточные объёмы перевалки на $4\%$, а как следствие и выручку порта.

Одна из ведущих горнодобывающих групп в СНГ на одном из принадлежащих ей угольных карьеров часто не выполняла планы по извлечению вскрышки, это приводило к срыву планов по добыче угля. Одно из проблемных мест --- нормативные данные по длительности операций устарели и не соответствовали реальным. Для создания модели карьера длительность всех операций замерялась секундомером. Благодаря точному моделированию работы угольный шахты в виртуальной среде стало возможным оценить потенциал наращивания производственных мощностей и свести к минимуму простой оборудования и техники из-за отсутсвия запчастей.

\subsection{Техническое задание}

Как было описано выше, одна из задач для создания цифрового двойника --- сбор и хранение данных, собираемых с системы. Необходиму решить эту задачу для последующего создания полноценного цифрового двойника и применения алгоритмов моделирования и машинного обучения.

Для работы системы мониторинга и предиктивных моделей с объектов предприятия собираются исходные данные. Оборудование оснащено датчиками, собирающими данные с разными частотами дискретезации.

Так как оборудование взаимосвязано и образует сложные технологические цепочки, границы принадлежности датчика к тому или иному оборудованию размыты. Один датчик может входить в состав моделей для разных единиц оборудования. Однако все датчики могут быть однозначно отнесены к одному объекту организационной структуры предприятия. Количество всех параметров для одного объекта может составлять несколько тысяч.

Задачей является разработать модуль, обеспечивающий управление структурой хранения временных рядов и данными сенсоров для системы мониторинга цифрового двойника промышленных электростанций, использующих газотурбинное оборудование. Временные ряды должны хранится в ClickHouse, а справочники в PostgreSQL. При реализации необходимо предусмотреть следующие особенности:

\begin{itemize}
    \item возможность определять структуру таблиц --- наборы и типы датчиков к привязке к оргнизационной структуре;
    \item частота дискретезации датчиков может быть разной: в какой-то момент времени не у всех датчиков есть значение, тогда это событие достраивается по последнему известному значению на этот момент;
    \item датчики имеют глобальные уникальные идентификаторы;
    \item для получения данных должна быть возможность получать вектора (все значения датчиков), временной ряд, набор рядов;
    \item механизм настройки, который будет позволять сопоставлять код датчика к организационной единице;
    \item дерево оборудования связано с сигналами отношением многие ко многим.
\end{itemize}