\conclusion

В результате выполнения выпускной квалификационной работы бакалавра был разработан модуль управления временными рядами сигналов сложных технических систем на языке Python с использованием СУБД PostgreSQL и ClickHouse. Были решены следующие задачи:

\begin{itemize}
    \item спроектирована модель данных дерева организационной структуры предприятия;
    \item описаны способы взаимодействия: добавление, удаление и изменение вершин и рёбер дерева;
    \item спроектирована модель хранения временных рядов датчиков;
    \item изучены средства и технологии, которые будут применятся в ходе разработки программного продукта;
    \item реализован модуль управления графом организационной структурой и данными;
    \item разработан алгоритм объединения данных датчиков с разными частотами дискретизации;
    \item реализована генерация данных для таблиц датчиков, алгоритм получения наборов временных рядов;
    \item произведён тест производительности реализованного модуля, проведено сравнение двух алгоритмов хранения и считывания данных.
\end{itemize}

Базы данных хранятся в постоянной памяти компьютера. Твердотельные накопители быстрее записывают и считывают данные, что может ускорить работу модуля. Для повышения надёжности рекомендуется создавать резервные копии баз данных, сохраняя их на жёсткий диск.

Созданный модуль позволяет изменить структуру дерева организационной структуры на предприятии, добавлять и удалять новые организационные единицы и датчики, изменять их параметры. Удалённые объекты хранятся в базе данных, чтобы была возможность сформировать список изменений.

При получении набора временных рядов датчиков с разными частотами дискретизации неизвестные значения интерполируются по последним известным.

Модуль автоматизирует сбор информации с сенсоров системы, тем самым упрощая создание цифрового двойника электростанции. Данные с датчиков надёжно хранятся в базе данных и будут использованы для моделирования объекта и предиктивной аналитики. Так можно будет оптимизировать работу оборудования, уменьшая скорость износа и повышая отказоустойчивость как отдельной электростанции, так и всей электросети в целом.
